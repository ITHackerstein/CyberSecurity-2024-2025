%! TeX program = lualatex

% TODO: Inserisci timeline della vulnerabilità

\documentclass[a4paper,oneside,12pt]{report}
\usepackage[utf8]{inputenc}
\usepackage[italian]{babel}
\usepackage{hyperref}
\usepackage{parskip}

\linespread{1.25}

\title{\textbf{CyberSecurity 2024-25} \\ CVE-2025-29927 in \texttt{Next.js}}
\author{Davide Carella \& Luca Ardito}
\date{}

\begin{document}

\maketitle
\tableofcontents
\newpage

\addcontentsline{toc}{chapter}{Introduzione}
\chapter*{Introduzione}
\label{chap:introduzione}

Nel contesto dello sviluppo web moderno, la sicurezza delle applicazioni rappresenta una sfida critica e sempre più complessa. I framework frontend e full-stack, come \texttt{Next.js}, forniscono funzionalità avanzate per lo sviluppo rapido di applicazioni web dinamiche, ma allo stesso tempo introducono superfici di attacco potenziali che, se trascurate, possono compromettere l'intera applicazione e i dati degli utenti.

\texttt{Next.js} è un framework open source basato su \texttt{React}, progettato per la realizzazione di applicazioni web ad alte prestazioni. Offre funzionalità come il rendering lato server (SSR), la generazione statica (SSG), il supporto per l'internazionalizzazione, la gestione delle API e il caricamento dinamico dei componenti. È particolarmente apprezzato per la sua semplicità d’uso, la scalabilità e le ottimizzazioni integrate per SEO e performance.

Questo elaborato si concentra sull’analisi di una vulnerabilità recentemente identificata in esso: la \emph{CVE-2025-29927}. Tale vulnerabilità ha avuto un impatto significativo, rendendo possibile, in specifiche condizioni, l’accesso non autorizzato a risorse protette, con gravi implicazioni per la riservatezza e l'integrità dei dati.

L’obiettivo del progetto è duplice: da un lato descrivere dettagliatamente la vulnerabilità, la sua origine e il suo meccanismo di sfruttamento, dall’altro offrire una panoramica delle tecniche di mitigazione adottate dal team di sviluppo di \texttt{Next.js} per correggerla. In particolare, verranno presentati:
\begin{itemize}
  \item una spiegazione tecnica su cosa compromette la vulnerabilità;
  \item l’analisi dell’entry point dell’attacco;
  \item il calcolo del punteggio \emph{CVSS} (Common Vulnerability Scoring System);
  \item un approfondimento sull’origine e sulla causa radice della falla;
  \item due \textit{Proof of Concept} per dimostrarne lo sfruttamento pratico;
  \item le contromisure adottate e possibili \textit{workaround} temporanei.
\end{itemize}

\chapter{Analisi della CVE}
\label{chap:analisi-cve}

\section{Descrizione ufficiale della CVE}
\label{sec:descrizione-ufficiale-cve}

Dal \href{https://cve.mitre.org/cgi-bin/cvename.cgi?name=CVE-2025-29927}{NIST NVD}:
\begin{quote}
``Next.js is a React framework for building full-stack web applications. Starting in version 1.11.4 and prior to versions 12.3.5, 13.5.9, 14.2.25, and 15.2.3, it is possible to bypass authorization checks within a Next.js application, if the authorization check occurs in middleware. If patching to a safe version is infeasible, it is recommend that you prevent external user requests which contain the x-middleware-subrequest header from reaching your Next.js application. This vulnerability is fixed in 12.3.5, 13.5.9, 14.2.25, and 15.2.3.''
\end{quote}

\section{Cosa compromette?}
\label{sec:cosa-compromette}

La vulnerabilità \emph{CVE-2025-29927} ha un impatto diretto su uno dei tre principi fondamentali della sicurezza informatica, comunemente noti come \emph{triade CIA}: \emph{Confidenzialità}, \emph{Integrità} e \emph{Disponibilità}.

\begin{itemize}
  \item \emph{Confidenzialità}: è il principio maggiormente compromesso da questa vulnerabilità. In presenza di configurazioni specifiche, un attaccante remoto non autenticato può accedere a risorse che dovrebbero essere protette da meccanismi di autenticazione o autorizzazione. Questo può includere pagine private, endpoint API interni o contenuti generati dinamicamente. La conseguenza è la potenziale esposizione di informazioni sensibili, con implicazioni gravi in termini di privacy e sicurezza.
  \item \emph{Integrità}: anche se in modo meno diretto, l’integrità può essere compromessa. Un attaccante che riesce a visualizzare informazioni riservate potrebbe sfruttarle per manipolare il comportamento dell'applicazione in una fase successiva. Ad esempio, accedendo a configurazioni interne o risposte API non destinate all’utente, potrebbe costruire richieste mirate in grado di alterare dati, forzare comportamenti non previsti o eludere meccanismi di controllo lato client. Sebbene la vulnerabilità non consenta di modificare direttamente i dati, può essere un punto di partenza per attacchi più avanzati che impattano l'integrità.
  \item \emph{Disponibilità}: non sono stati rilevati impatti sulla disponibilità dei servizi. La vulnerabilità non introduce vettori di attacco per causare interruzioni o rallentamenti del servizio (DoS), né limita l’accesso alle funzionalità legittime dell’applicazione.
\end{itemize}

In sintesi, la \emph{CVE-2025-29927} compromette principalmente la \emph{confidenzialità} dei dati e, in misura minore e solo in condizioni particolari, anche l’\emph{integrità}. La \emph{disponibilità}, invece, non viene generalmente alterata dall’exploit di questa vulnerabilità.

\section{Dove parte l'attacco?}
\label{sec:dove-parte-attacco}

L'attacco associato alla \emph{CVE-2025-29927} ha origine a livello del \emph{routing} di \texttt{Next.js}, in particolare nella gestione delle richieste da parte del \emph{middleware} applicativo. In alcune configurazioni, il middleware non applica correttamente le restrizioni di accesso previste per determinate route dinamiche o protette. Questo comportamento anomalo consente a un attaccante remoto non autenticato di accedere a contenuti o API destinati a utenti autenticati, semplicemente formulando una richiesta HTTP ben costruita verso l’endpoint vulnerabile.

Il punto d’ingresso principale è quindi una richiesta HTTP inviata verso un percorso protetto che, però, non viene intercettata correttamente dal sistema di protezione. In particolare, la vulnerabilità si manifesta quando l’header \texttt{x-middleware-subrequest} non viene gestito correttamente, oppure viene interpretato in modo non sicuro, causando una bypass del middleware che normalmente dovrebbe bloccare l’accesso non autorizzato.

L’attaccante non ha bisogno di interagire con l’interfaccia utente dell’applicazione: è sufficiente che conosca o intuisca gli endpoint sensibili (ad esempio, \texttt{/dashboard}, \texttt{/admin}, ecc.) e che invii una richiesta manipolata direttamente dal browser, da uno script o tramite uno strumento come \textit{curl} o \textit{Burp Suite}.

La facilità con cui si può identificare e sfruttare il punto d’ingresso rende la vulnerabilità particolarmente pericolosa, soprattutto in ambienti dove le route riservate sono prevedibili o mal configurate.

\section{Come viene calcolato il CVSS?}
\label{sec:come-viene-calcolato-cvss}

La versione utilizzata al momento della pubblicazione della \emph{CVE-2025-29927} è la \emph{CVSS v3.1}, che calcola un punteggio finale compreso tra 0.0 (nessun impatto) e 10.0 (massima gravità), suddiviso in tre gruppi di metriche.

Uno score per la vulnerabilit\`a \`e stato gi\`a calcolato con i seguenti parametri:
\begin{itemize}
  \item \emph{Metriche Base}:
	\begin{itemize}
		\item \texttt{AV:N} -- l'attacco è eseguibile da remoto via rete;
		\item \texttt{AC:L} -- l’exploit non richiede condizioni particolari;
		\item \texttt{PR:N}: -- l’attaccante non deve essere autenticato;
		\item \texttt{UI:N} -- l’attacco non richiede interazione da parte dell’utente;
		\item \texttt{S:U} -- l’impatto resta confinato nel contesto dell’applicazione vulnerabile;
		\item \texttt{C:H}: -- possibile accesso a dati sensibili;
		\item \texttt{I:N}: possibile modifica di dati sensibili;
		\item \texttt{A:N}: -- non si verificano impatti sulla disponibilit\`a.
	\end{itemize}
  \item \emph{Metriche Temporali e Ambientali}: non disponibili.
\end{itemize}

Sulla base di queste metriche, la \emph{CVE-2025-29927} riceve un punteggio base \emph{CVSS 3.1} di 9.1.

\chapter{Analisi della causa radice}
\label{chap:analisi-causa-radice}

% TODO: Leggi https://zhero-web-sec.github.io/research-and-things/nextjs-and-the-corrupt-middleware

\section{Origine della vulnerabilit\`a}
\label{sec:origine-vulnerabilita}

\chapter{Exploitation}
\label{chap:exploitation}

\section{Fasi dell'attacco}
\label{sec:fasi-attacco}

\section{\textit{Proof of Concept} (PoC) in Python}
\label{sec:poc-python}

\section{\textit{Proof of Concept} (PoC) in Burp Suite}
\label{sec:poc-burp-suite}

\chapter{Fixes e Mitigazione}
\label{chap:fixes-mitigazione}

\section{Aggiornamento del framework}
\label{sec:aggiornamento-framework}

\section{\textit{Workaround} momentaneo}
\label{sec:workaround-momentaneo}

\addcontentsline{toc}{chapter}{Conclusioni}
\chapter*{Conclusioni}
\label{chap:conclusioni}

\addcontentsline{toc}{chapter}{Riferimenti}
\chapter*{Riferimenti}
\label{chap:references}

\end{document}

